\documentclass[12pt]{article} 

\usepackage[utf8]{inputenc}    
\usepackage[T1]{fontenc}       
\usepackage{lmodern}           
\usepackage{amsmath, amssymb}  
\usepackage{graphicx}          
\usepackage{geometry}          
\usepackage{hyperref}          
\usepackage{fancyhdr}          
\usepackage{parskip}           
\usepackage{tikz}              
\usepackage{booktabs}          
\usepackage{enumitem}          
\usepackage{caption}           
\usepackage{listings}          
\usepackage{multirow}    
\usepackage{xcolor}      
\usetikzlibrary{automata, positioning}
\lstset{
  frame=tb,
  language=C,
  aboveskip=3mm,
  belowskip=3mm,
  showstringspaces=false,
  columns=flexible,
  basicstyle={\small\ttfamily},
  numbers=none,
  numberstyle=\tiny\color{gray},
  keywordstyle=\color{blue},
  commentstyle=\color{brown},
  stringstyle=\color{orange},
  breaklines=true,
  breakatwhitespace=true,
  tabsize=3
}
\hypersetup{
    colorlinks=true,
    linkcolor=blue,
    filecolor=magenta,      
    urlcolor=cyan,
    pdfpagemode=FullScreen,
    }

\geometry{margin=1in} 


\pagestyle{fancy}
\fancyhf{}
\fancyhead[L]{\leftmark} 
\fancyhead[R]{\thepage}  
\newtheorem{Correction}{Correction}

\title{CS3231 Tutorial 2}
\author{WANG Xiyu}
\date{\today}

\begin{document}

\maketitle

\tableofcontents 

\section{}

\section*{Prove or disprove}
\subsubsection*{}
\[L((R + S)^*) = L((R^*S^*)^*)\]
\subsubsection*{Solution:}
Suppose string $w \in L(R+S)$, $w = t_1t_2t_3...t_n$, for some $n \in \mathbb{N}, t_i \in R \lor t_i \in S$.
Which fits the definition of $L(R^*S^*)$ since both accept arbituary combination of $s \in S$ and $r \in R$, or arbituary length $n \in \mathbb{N}$.
\newline
Therefore, $L(R+S) \subseteq L(R^*S^*)$ 
\newline
\newline
Now suppose string $u \in L(R^*S^*)$, $u = k_1k_2k_3...k_n$, for some $n \in \mathbb{N}$, $k_i \in R^* \lor k_i \in S^*$.
Which fits the definition of $L(R+S)$ for the same reason above.



\subsubsection*{}
\[L(S(R+S)^*S) = L((SR^*S)^+)\]
\subsubsection*{Solution:}
 
\end{document}
